% !Mode:: "TeX:UTF-8" 
\begin{conclusions}

在本项目的结论中,我们首先概括论文的主要内容和贡献,然后对存在的不足之处提出展望。

单人姿态估计是视觉领域难度极大的研究方向之一。但在应用场景上,单人姿态估计表现出了极大的需求,虚拟试衣、智能安防、体育健身、VR技术等场景都亟需该算法的落地应用。

人机交互的主要目的是使机器人能够学习和了解人,能领会和模仿人的语言和行为。为了是人机交互自然,必须引入类似于人与人之间的沟通方式,即依赖语音与视觉。在这种背景下,人体姿态估计在人机交互方面有着举足轻重的作用。本文就是对此提出了基于人体姿态估计的机器人姿态跟踪算法,系统地完成了足够轻量化的人体姿态估计算法,并落实到了移动端,机器人端。具体的研究内容和研究成果如下:

\begin{enumerate}
\item 系统全面地介绍了人体姿态估计所涉及领域的主要内容,包括卷积神经网络的组成、反向传播算法、激活函数以及正则化。并且介绍了国内外对于单人姿态估计算法的发展历程以及各自的优势,为后续的BlazePose算法研究打下基础。

\item 研究了BlazePose算法,将目前最主流的基于热图的姿态估计和基于回归的姿态估计相结合。使用编码器-解码器网络架构来预测所有关节的热图。同时后面跟着另一个编码器,直接回归到所有关节的坐标。值得注意的是,热图分支可以在推理过程中被丢弃,使得网络足够轻量化,使得其可以架构在移动端和机器人端。

\item 在目标检测的后处理步骤中,不使用NMS算法,而是假设人脸这一相对刚性的目标一直会出现。因为多个不明确的框满足NMS算法的交并集(IoU)阈值,而人脸检测器则打破了这一限制。

\item 同时我们拓展了现在比较基础的17的关键点,增加至33个关键点,使得模型所表达的语义信息更加丰富,以便于可以用到更复杂的应用场景。

\item 然后,我们还对比了重构的模型与官方模型以及其它模型的性能,结果显示,重构模型的准确率能达到82.5\%,并且在帧数上也远超OpenPose接近官方模型。

\item 最后,我们一步步将模型部署到电脑端,移动端,虚拟机器人以及真实机器人上。从时间同步性以及动作准确度的角度来说,该模型已经达到实用的程度。
\end{enumerate}

本次研究复现的模型虽然达到了一定的效果,但依然存在一些局限性,结合市场需求以及国内外目前研究内容,对该模型作出如下展望:

\begin{enumerate}
\item 进一步提高模型的轻量化和性能,使得模型能在部署在性能较差的机器人上,并且使得机器人处理更加复杂的动作,改善交互体验。

\item 该模型是2.5D模型,虽然拥有着Z轴的分量,但是语义信息的损失极大,如何在保证模型轻量化的同时提高Z轴分量信息的准确度是一大难题。

\item 人体动作关键帧的准确提取。机器人通过自身的视觉系统捕获人体动作序列帧时,不用对每一帧图像进行处理,只需处理关键帧,其余的帧通过插值的方法估计出即可。如何得到这些动作关键帧需要进一步研究。
\end{enumerate}

\end{conclusions}
